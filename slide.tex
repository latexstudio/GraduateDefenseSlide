\documentclass{beamer}
\usepackage[UTF8]{ctex}

\usetheme{metropolis}
\setbeamerfont{title}{size=\zihao 4}
\usepackage{appendixnumberbeamer}
\usepackage{amsmath}
\usepackage{enumerate}


\title{基于Spatial-Spark网络空间海量数据分析与挖掘}
\author[高峰]{
    \makebox[2.5em][s]{姓名:} \makebox[3em][s]{高峰}\\
    \makebox[2.5em][s]{导师:} \makebox[3em][s]{高井祥} \\
    \makebox[2.5em][s]{} \makebox[3em]{孙久运} \\
    \makebox[2.5em][s]{专业:} \makebox[10em][l]{大地测量学与测量工程}\\
}
\date{}
\titlegraphic{\hfill\includegraphics[height=1.5cm]{figures/cumt.pdf}}

\begin{document}
% make title
\maketitle

% make content
\begin{frame}{提纲}
  \setbeamertemplate{section in toc}[sections numbered]
  \tableofcontents[hideallsubsections]
\end{frame}

\section{绪论}

\begin{frame}{绪论}
    \begin{columns}
        \begin{column}{0.5\textwidth}
        \includegraphics[scale=0.3]{figures/smartcity.jpg}
        \end{column}

        \begin{column}{0.5\textwidth}
        \textbf{智慧城市:} 
        
        数字城市、物联网、云计算
        % \includegraphics[scale=0.2]{figures/smartcity.jpg}
        \vspace{2em}

        \pause
        \alert{数字城市}
        \begin{itemize}
        \pause
        \item 空间信息快速获取技术
        \pause
        \item 海量空间数据管理
        \pause
        \item 空间信息可视化技术
        \pause
        \item 空间数据分析挖掘和网络服务技术
        \end{itemize}
        \end{column}
   \end{columns}
\end{frame}

\begin{frame}{绪论}
    \textbf{移动互联网发展(LBS)}

    打车软件、O2O软件、社交网络软件

    \vspace{2em}
    \pause
    \textbf{意义}
    \begin{itemize}
        \pause
        \item 对个人而言,对社交空间数据挖掘,可以发现不同的知识,为生活带来 便利
        \pause
        \item 对商业公司而言,通过空间数据挖掘发现目标客户使用习惯或群体性活 动规律,对商业推广计划提出指导性意见
        \pause
        \item 对政府决策部门而言,能够发现社会的经济、文化、交通等众多方面活动规律
    \end{itemize}
\end{frame}


\section{相关技术}
\section{Spatial-Spark}
\section{POI空间分析}
\section{人口流动网络分析}
\section{总结}
\section*{参考文献}
\end{document}